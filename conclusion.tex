\chapter*{Conclusion}
\addcontentsline{toc}{chapter}{Conclusion}
\markboth{Conclusion}{Conclusion}

The objective of our thesis was to study the character of errors
of the approximate histograms, and to determine whether these histograms
can be used for sufficiently precise genome size estimates.

\medskip
We focused on software Kmerlight \cite{Sivadasan2016}, since it 
is, to our knowledge, the fastest and most memory efficient algorithm for
$k$-mer abundance histogram approximation. 

From the experimental observations, we discovered that Kmerlight's estimates 
of $\hat f_i$ are biased towards higher values for histogram columns
with low $f_i / F_0$ ratio. We identified the source of bias: Kmerlight
chooses levels $w_i^*$ that maximize $t_i^{(w)}$, the number of collision-free
counters with value $i$; and we proposed and tested a modification which successfully
eliminated the bias under the assumption that all hashing collisions can be detected. 
Our modified version of Kmerlight uses only one level $w^+$ which
maximizes $E(t_i^{(w)})$. To select $w^+$ we only need an estimate of $F_0$ and therefore,
we believe that a more memory-efficient method of histogram approximation can be devised by
further improvement of Kmerlight. 

As we could compute histograms with zero mean estimation error $E(\hat f_i - f_i) = 0$,
we focused on the variance of approximation errors. We were able to precisely model the
distribution of Kmerlight's errors with normal distribution for sufficiently high values of
$f_i$ and we produced a quantitative estimate of errors' variance for all reasonable values of $f_i$
(those $f_i$ for which $E(t_i^{(w^+)}) > 1$). 

Using the estimate of errors, we proposed a method
for selecting suitable parameters for Kmerlight, and generally, we believe that the 
ability to estimate the extent of Kmerlight's errors without the actual histogram computation, 
makes a more practical tool.
Since Kmerlight can be generalized to approximate an abundance histogram of any items,
not only of $k$-mers, the applicability of our work is thus not limited to the
bioinformatics context. 

% In our work we also analyzed the errors of KmerGenie, though the analysis was simple.
% There is also one more algorithm which approximates the histogram that we know of: khmer,
% devised by Zhang et al. \cite{Zhang2014}. We did not study khmer, so the topic of khmer's errors 
% and further use of its histograms might become a subject of future work.

\medskip

With the first part of our objective successfully fulfilled, we turned our attention towards
the genome size estimation and CovEst software. To investigate the precision of genome size 
estimates based on approximated histograms, we performed experiments on simulated genomes.

We examined the influence of genome parameters on the accuracy of coverage estimates.
With increasing error rate and coverage, the number of unique $k$-mers increases
and thus the histogram approximations produced by Kmerlight are less precise. Therefore
also the estimates of the coverage are less precise. The genome size does not seem to 
significantly influence the relative precision of coverage estimates based on Kmerlight's
histograms. Since the ratios of $f_i/F_0$ change only slightly with
various genome sizes, relative errors of Kmerlight also do not change much.

Our experimental evaluation showed that even with approximated histograms, an estimation error
of less than 1\% can be achieved for most usual datasets (sets of reads with $c > 2$ 
and $e < 0.05$), and also that estimates based on modified Kmerlight's histograms achieve 
lower mean errors than the estimates of the original Kmerlight. 

We note that we did not test CovEst estimates on real genomes, but only on simulated genomes. 
A future work may clarify how other biological phenomena influence CovEst's estimates. 

% With aim to increase CovEst's precision on approximated histograms,
% we explored different methods that can be used to compare the guessed histogram to the observed
% one. Even though we were not able to increase CovEst's precision on histograms produced
% by Kmerlight and KmerGenie, we clarified why CovEst is robust with respect to these histograms.
