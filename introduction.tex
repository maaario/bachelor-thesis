\chapter*{Introduction}
\addcontentsline{toc}{chapter}{Introduction}
\markboth{Introduction}{Introduction}

In a DNA sequencing experiment, many reads are produced from a genome.
These reads are short strings obtained from random locations of the genome,
potentially with some sequencing errors.
For a genome to be analyzed, the genomic sequence must be first assembled from the reads,
but the process of genome assembly is computationally demanding.

In previous years, several methods were devised to estimate
the size of a genome and some other genome characteristics
without the need of genome assembly \cite{Hozza2015, Williams2013, Melsted2014,
Sivadasan2016}. To compute these estimates,
a summary statistic of reads, $k$-mer abundance histogram is used.
In order to produce such a histogram, the sequencing reads are first processed into $k$-mers 
(substrings of a length $k$). Then the histogram of the number of occurrences of $k$-mers
can be computed. 

Most of the histogram computing methods are based on hash tables or suffix arrays 
\cite{Melsted2011, Marcais2011, Rizk2013, Kurtz2008} and
their memory usage increases at least linearly with the number of processed distinct $k$-mers. 
To reduce the memory requirements, $k$-mer abundance histograms can be computed approximately.
One of the newest such algorithms, Kmerlight \cite{Sivadasan2016}, combines the
techniques of sampling and hashing to maintain a sketch of $k$-mers 
and from the contents of the sketch computes an estimate of the histogram.

The approximate histograms were already used as an inputs for genome size estimation tools,
however the impact of the approximation errors on estimate precision was not evaluated.

\medskip

The goal of our thesis is to study the character of errors of the approximate histograms
and their influence on the subsequent genome size estimation.

We start with an empirical study of the approximation errors of Kmerlight algorithm, and
we discover that Kmerlight produces systematically biased estimates of some histograms.
We explain the source of the bias and mathematically support our claims, and then
we also propose a modification of Kmerlight which eliminates this bias.

Next we model the distribution of Kmerlight's errors with the normal distribution,
and we propose a formula that describes Kmerlight's variance. We then experimentally test
our theoretical model and explore its limitations.

Finally, we use CovEst software \cite{Hozza2015} to estimate the sizes of simulated
from approximate histograms produced by Kmerlight. We describe how different parameters of
the genome influence the accuracy of the estimates and we compare the estimates
based on the exact histograms to the estimates based on approximate histograms.

\medskip

In the first chapter we explain the essential biological vocabulary and
we outline the principles of histogram computing and of genome size estimation.
In the second chapter we study the approximation errors made by Kmerlight
and in the last chapter we investigate the accuracy of genome size estimates
based on approximate histograms.
