\section{Problem Statement}
The main goal of our thesis is to determine whether the approximate histograms
produced by Kmerlight or KmerGenie can be used as inputs for CovEst to
estimate the genome size. We mainly focus on preserving the 
accuracy of CovEst's estimates, and we focus less on the processing time and memory consumption. 

In order to achieve our goal, we study the performance of CovEst on these approximate
histograms and look for ways to increase CovEst's accuracy. Two approaches
come into mind: either we alter the histogram counting algorithm and the approximate
histogram to better suit the needs of CovEst, or we alter the probabilistic model of CovEst to
cope with imprecise histograms. We explore both approaches. 

In chapter \ref{sec:chapter2} we study the qualitative and quantitative character
of inaccuracies introduced by Kmerlight\footnote{The analysis of KmerGenie's errors is trival
and it will be presented later in section \ref{sec:kmergenie-errors}.} with the aim to model or
predict the distribution of the estimates of $f_i$. We discover that Kmerlight can be modified
to produce more accurate histograms.

In chapter \ref{sec:chapter3} we turn our attention to CovEst. We find that CovEst is robust
with respect to Kmerlight's and KmerGenie's errors, and that we are unable to further increase
its accuracy on approximated histograms. Nevertheless, we present a new method which helps CovEst
with a different type of histogram errors that might be used in the future research.
