\section{Problem Statement}
The main goal of our thesis is to determine whether the approximate histograms
produced by Kmerlight can be used as inputs for CovEst to
estimate the genome size. We mainly focus on preserving the 
accuracy of CovEst's estimates, and we focus less on the processing time and memory consumption. 

In order to achieve our goal, in chapter \ref{sec:chapter2} we study the qualitative
and quantitative character of inaccuracies introduced by Kmerlight with the aim to model or
predict the distribution of the estimates of $f_i$. We discover that Kmerlight can be modified
to produce more accurate histograms and we present a model of Kmerlight's errors distribution.

In chapter \ref{sec:chapter3} we we study the performance of CovEst on approximate
histograms. After providing a more detailed description of CovEst software
we experimentally evaluate the effects of various coverages, error rates and genome sizes
on precision of genome size estimates. We find that CovEst is robust with respect to 
Kmerlight's errors and that CovEst's estimates based on approximate histograms maintain
a sufficient precision.
