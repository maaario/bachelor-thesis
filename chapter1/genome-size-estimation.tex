\section{Genome Size Estimation}
\label{sec:intro-estimation}
As we mentioned in section \ref{sec:assembly}, it would be helpful to know the length of a genome 
before its assembly. Furthermore, we might also be interested in other genome characteristics,
such as the extent and multiplicity of its duplicated regions.

In previous years there were several researches concerned with the estimation 
of these characteristics without the need of computationally demanding 
genome assembly \cite{Hozza2015, Williams2013, Melsted2014, Sivadasan2016}, using only the $k$-mer 
abundance histogram.

\medskip

Let us summarize the parameters of the genome in Table \ref{tab:genome-parameters}.
Then we will describe how different genome characteristics influence the shape of the histogram,
and how these parameters can be estimated from the histogram. We use the assumption that
the reads are distributed uniformly across the genome. 

\begin{table}[h!]
\centering
\begin{tabular}{ l l l }
 $L$ & genome size & the number of nucleotides in the genome \\  
 $l$ & read length & the number of nucleotides in one read \\
 $c$ & coverage & an average number of reads covering each genome position \\ 
 $e$ & error rate & probability that a single nucleotide was read incorrectly   
\end{tabular}
\caption[Genome parameters]{A set of simple genome parameters as described in the section \ref{sec:sequencing}.
\label{tab:genome-parameters}}
\end{table}

When we consider a genome without duplicated sequences and sequencing without errors, we
would expect the histogram to reach its maximum close to the abundance $c$.
Under the assumption that the reads are uniformly distributed over the genome,
most of the locations in genome are expected to be covered by approximately $c$ reads,
thus contributing to values close to $f_c$.

Note that the average coverage is a fraction of the sum of lengths of all reads
and the genome size ($c = n \cdot l/L$, if $n$ denotes the number of all reads). 
Since the sum of lengths of reads is known, with the knowledge of the coverage, we can estimate the 
genome size, and conversely, with the knowledge of the genome size we can estimate the coverage.

\medskip

If the average coverage of the whole genome is $c$, then a duplicated sequence is expected 
to be a part of twice as many reads as a unique sequence, and so to have coverage $2c$. 
A sequence copied three times in the genome is expected to have coverage $3c$.
The number of copies present can be thus deduced from the positions of peaks in the histogram,
expected at abundances $2c, 3c, \dots nc$ and the extents of copied regions can be deduced
from the relative heights of these peaks $f_{2c}, f_{3c}, \dots f_{nc}$.

\medskip
 
The sequencing error rates also change the shape of the histogram. With error rate of $1\%$ 
and $k=21$, about $19\%$ of $k$-mer occurrences are erroneous and thus unique with high probability, 
creating a notable peak in the histogram at value one. From the height of $f_1$, the error rate
can be estimated. 

Furthermore, the erroneous $k$-mers also decrease the average effective coverage,
as some $k$-mers covering each location are wasted because of errors. As a result, the whole
histogram is shifted towards lower abundances.

\medskip

The aforementioned characteristics lay out the basis of $k$-mer abundance histogram analysis.
Approaches introduced in the previous work \cite{Hozza2015, Williams2013, Melsted2014,
Sivadasan2016} use probabilistic generative models that can generate the
expected shape of the histogram based on several parameters. These parameters are chosen
and optimized to produce a histogram most alike to the observed one.
We will provide a more detailed description of one such model, CovEst \cite{Hozza2015, Hozza2016},
in section \ref{sec:covest}.
