\chapter{Use of Histograms for Genome Size Estimation}
\label{sec:chapter3}
In the beginnning of this chapter we provide a more detailed description of a probabilistic model
that can be used to estimate genome characteristics using only $k$-mer abundance histogram.
This model was presented in \cite{Hozza2015, Hozza2016} and was implemented in a software CovEst.

Next we present our work. We study the the accuracy of CovEst estimates based on
approximate histograms and we improve the accuracy by using our new modified version of Kmerlight.

In the end of this chapter we explore a different way to improve CovEst's accuracy. 
We conclude that this method does not improve CovEst's estimates on histograms produced by
Kmerlight and KmerGenie, but can be used to make CovEst tolerant to other types of
approximate histograms.

\section{CovEst}
\label{sec:covest}
CovEst uses a $k$-mer abundance histogram $f = f_1, f_2, \dots, f_m$ as its input,
and finds the parameters $\theta = (c, e)$\footnote{Coverage and error rate, as described 
in the introductory section \ref{sec:intro-estimation}. In a more advanced version, $\theta$
also includes parameters related to the extent of multiplied regions.} that maximize
the likelihood of the observed histogram $L(\theta) = P(f \,|\, \theta)$.

Using the paramaters $\theta$, CovEst, in fact, first generates the expected histogram
shape $p_1, p_2, \dots p_m$, where $p_i$ denotes the probability that a distinct $k$-mer
occurs with abundance $i$. Then the likelihood of this shape is evaluated and thus 
two sets of different parameters $\theta, \theta'$ can be compared.

\paragraph{Generative model}
While the average coverage of a genome nucleotide is $c$, the average coverage
of a $k$-mer is lower, since to produce a $k$-mer, a read must cover the whole
$k$-mer. CovEst thus internally uses a value $c_k = (r-k+1)/r \cdot c$ to describe the
average coverage of a $k$-mer. This is the only assumption made which involves the reads.

In the following text we will consider the $k$-mers to be uniformly distributed
over the genome. This assumption enabled the authors to model the abundance of a $k$-mer
with a Poisson distribution with mean $c_k$. Since $k$-mers with abundance zero are never
observed, it is important to use a zero-truncated Poission distribution, 
which leads to the following formula for $p_i$:
% As the probability of obtaining the value greater than zero is 
% $(1 - e^{c_k})$ in the Poisson distribution, we renormalize $p_i$ with this term:

$$ p_i = Poi_0(i; c_k) = \frac{c_k^i e^{-c_k}}{i!(1 - e^{-c_k})} $$

The authors assume that at sequencing, the probability that a nucleotide will be read
correctly is $1-\eps$ and the probability that it will be read as a specific different
nucleotide is $\eps/3$. Therefore, if the Hamming distance of two $k$-mers is $s$, the
probability that one was obtained by reading another one is $(\eps/3)^s(1-\eps)^{k-s}$.
The distribution of the $k$-mers produced by $s$ errors can be
therefore modelled with Poisson distribution with mean 
$\lambda_s = (\eps/3)^s(1-\eps)^{k-s} \cdot c_k$ and to compute $p_i$ we would use
a weighed sum of these distributions:
$$ p_i = \sum_{s=0}^k \alpha_s \cdot Poi_0(i; \lambda_s) $$
where the coefficients $\alpha_s$ sum to one ($\sum_{s=0}^k \alpha_s = 1$) and they correspond
to fractions of $k$-mers produced by $s$ errors. From one $k$-mer ${k \choose s} 3^s$ 
different $k$-mers can be obtained as a product of $s$ errors and the probability, 
that at least one of these $k$-mers will be observed is $(1 - e^{-\lambda_s})$. 
Therefore the coefficients $\alpha_s$ will be set to have values corresponding to 
$\alpha_s \propto {k \choose s} 3^s(1 - e^{-\lambda_s})$.

The authors also modelled the repeats by adding together the contributions made by
$k$-mers that were copied once, twice\dots which can be again expressed as a weighed sum of
distributions. We will not describe the repeat modelling, since we will not use this extension
in our thesis, but the details can be found in the orignal work \cite{Hozza2015, Hozza2016}.  

%The authors further assume that each distinct $k$-mer comes only from one source $k$-mer.
%In other words, all occurences of the histogram $k$-mer $y$ were produced by
%reading a single $k$-mer $x$ in the genome and also that $k$-mer $x$ did not produce any
%other different $k$-mers as a result of errors.

\paragraph{Likelihood computation}
To evaluate the validity of the parameters $\theta$, the generated shape of the histogram is
compared to the observed histogram by likelihood computation
$L(\theta) = P(f\,|\,\theta) = P(f_1, f_2, \dots f_m\,|\,p_1, p_2, \dots, p_m)$. In order to
compare two histograms, the authors assume that each distinct $k$-mer is sampled
from a multinomial distribution $Mult(F_0, p_1, p_2, \dots p_m)$ and thus the
likelihood of the observerved histogram is 
$\frac{F_0!}{f_1! \cdot f_2! \cdot \dots f_m!} p_1^{f_1} p_2^{f_2} \dots p_m ^{f_m}$.
Since we are interested only in maximizing the likelihood, log-likelihood is used and
the constant terms are omitted, leaving the following expression to be maximized:
\begin{equation} \label{eq:covest-likelihood}
l(\theta) = \sum_{i=1}^m f_i \log p_i
\end{equation}
Note that the selection of this method to compare two histograms is arbitrary.

\paragraph{Parameters optimization}
As a starting point for parameter optimization, CovEst uses a very simple model
to produce an initial guess of $\theta$. This model assumes that every
erroneous $k$-mer has abundance one and thus the histogram shape can be modelled
as a mixture of two distributions of erroneous and error-free $k$-mers. The
simplicity of this model allows us to compute the optimal $c, e$ only by the Newton method.

Then a standard optimization algorithm L-BFGS-B is used to find the parameters $\theta$
that maximize the likelihood of the observed histogram, using the full model and the
likelihood computation explained above.

\paragraph{Exerimental evaluation}
The authors evaluated the precision of CovEst on generated genomes \cite{Hozza2015}
and concluded that CovEst can estimate the coverage (and thus the genome size) within 1\% 
of the true value for the common parameters. Moreover, CovEst produces reasonable
estimates even for the datasets with coverage less than one or with error rate of $10\%$.
