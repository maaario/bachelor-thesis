\section{Alternatives to Likelihood Computation}
The approximate histograms constitute a new set of inputs for CovEst.
With an ambition to make CovEst more precise and more robust to the approximation inaccuracies,
we would like to alter CovEst to be able to cope with the inacuraccies introduced by
Kmerlight or KmerGenie.

As we show in \ref{sec:estimation-variance}, we can estimate the variance and 
model the disribution of Kmerlight's errors. We noticed that different histogram columns $f_i$
have different variances and so we want to incorporate these differences into CovEst.

When CovEst searches for the most likely parameters $\theta$, it evaluates the likelihood
of the observed histogram by comparing $\hat f_1, \hat f_2, \dots, \hat f_m$ and 
$p_1, p_2, \dots p_m$. The idea of the improvement is to provide the variances of 
$\hat f_i$ as an additional input to CovEst. With these variances, CovEst could
attribute a higher importance to the columns with lower variance and attribute a 
lower importance the the columns with value $\hat f_i$ fluctuating more.

Let us denote the true histogram as $f$ and the approximated histogram as $\hat f$.
CovEst should seek the shape of histogram $f$ by maximizing $P(\hat f \,|\, f)$. 

\subsection{KmerGenie's Errors}
\label{sec:kmergenie-errors}
In \ref{sec:simple-sampling} we described a method used by KmerGenie to compute 
an approximate histogram. In this section we provide a simple analysis of KmerGenie's
errors that will in turn help us understand the relationship of Kmerlight's histograms
and CovEst.

Each distinct $k$-mer is either counted by KmerGenie with probability $1/s$ or it is discarded 
with probability $1 - 1/s$. Therefore, the abundance of a $k$-mer does not change 
with sampling process -- either all of its occurences are counted or none of them are.
This means that the estimate $\hat f_i$ is computed only from those $f_i$ distinct $k$-mers 
that contribute to the column $f_i$ by having abundance $i$. No other $k$-mers can
influence the value $\hat f_i$.

On average $f_i / s$ $k$-mers with abundance $i$ are chosen and counted by KmerGenie.
Moreover, the number of these $k$-mers follow a binomial distribution $Bin(f_i, 1/s)$
and it holds for the estimates that
$$\hat f_i \sim s \cdot Bin(f_i, 1/s)$$
The random vector $(\hat f_1, \hat f_2, \dots, \hat f_m, d)$, where $d$ is a number
of distinct discarded $k$-mers follows a multinomial distribution 
$$s \cdot Mult\left(F_0, \frac{f_1}{F_0} \frac{1}{s}, \frac{f_2}{F_0} \frac{1}{s}, \dots, 
\frac{f_m}{F_0} \frac{1}{s}, 1-\frac{1}{s}\right)$$
that can be interpreted as putting $F_0$ $k$-mers into $m+1$ buckets. With this knowledge we can
calculate the probability $P(\hat f\,|\,f)$.

\medskip

How to incorporate this knowledge into CovEst? We can notice that, in fact, to maximize 
$P(\hat f\,|\,f)$ it is sufficient to find $\arg \max_f \left(\sum_{i=1}^m \hat f_i \log f_i
\right)$. And this is exactly the same expression, that CovEst maximizes.

Here we find that CovEst by design expects the inaccuracies produced by KmerGenie
and thus there is no space for improvement.

\subsection{Likelihood Function for Kmerlight}

\subsection{Minimizing the Sum of Square Errors}

