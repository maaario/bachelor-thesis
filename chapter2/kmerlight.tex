\chapter{Analysis and Improvement of Kmerlight}
\label{sec:chapter2}
In this chapter we first present a detailed description of an exising algorithm
Kmerlight \cite{Sivadasan2016} which computes an approximated $k$-mer abundance histogram.

Aftewards we investigate the character of inaccuracies of the estimated histogram, and we present
a novel estimate of error variance. As we discover a systematic estimation bias, we alter
Kmerlight to produce unbiased histogram estimates. 

\section{Kmerlight}
\label{sec:kmerlight}
The input expected by Kmerlight is a collection of $ACGT$ sequences (reads).
Kmerlight transforms reads into $k$-mers, as it was described in \ref{sec:histogram}
and then processes a stream of $k$-mers. Kmerlight maintains a sketch of previously
processed $k$-mers and updates the sketch with each $k$-mer. The estimates 
of $F_0$ ($F_0 = \sum_{i=1}^{m} f_i$) and $f_i$ are computed from the content of
the sketch in the end. The output consists of values $\hat F_0, \hat f_1, 
\hat f_2, \dots, \hat f_m$.

Kmerlight's sketch consists of $W=64$ levels. There is a hash table $T_w$ at each level 
$w$ with $r$ counters $T_w[0], T_w[1], \dots, T_w[r-1]$.
Each counter $c$ stores its value $T_w[c].v$ (the number of elements stored in the counter)
and an auxiliary information $T_w[c].p$ from within a range $0, \dots, u-1$.


\paragraph{Update}
\begin{itemize}
\item For a distinct $k$-mer, its level is selected so that the probablity of selecting 
level $w$ is $1/2^w$. In particular, the $k$-mer is hashed into an integer of $W$ bits 
and the number of trailing zeroes determines the level $w$. Thus all occurences of the same
$k$-mer will be placed to the same level $w$.

\item Next, using a different hash function
$h: \{A, C, G, T\}^k \rightarrow \{ 0, \dots, r-1\} \times \{ 0, \dots, u-1\}$, 
the $k$-mer is hashed into a pair $(c, j)$. 
\begin{itemize}
\item If the counter $c$ at level $w$ is empty, its value is increased to 1 and $j$
is stored as an auxiliary information  $T_w[c].p$.
\item If the counter $T_w[c]$ is not empty, but the auxiliary information in $T_w[c].p$ 
is equal to $j$, the counter value $T_w[c].v$ is increased.
\item Finally, if $T_w[c].p \neq j$, the counter is marked as dirty with value -1.
Dirty counters are not modified by future updates.
\end{itemize}
\end{itemize}
Note that all occurences of the same $k$-mer will be stored in the same counter, 
and the value of the counter should correspond to the abundance of this $k$-mer.
Since two or more different $k$-mers may hash into the same counter at the same
level $T_w[c]$, collisions may occur. The auxiliary information helps to detect
some of these collisions.


\paragraph{Estimator of $F_0$}
Since on average $F_0 / 2^w$ distinct $k$-mers are hashed into level $w$, the probabilty
that a counter at level $w$ remains empty is approximately $p = (1 - \frac{1}{r})^{F_0/2^w}$.
In this estimate and in all subsequent analysis we assume that the number of distinct $k$-mers
at level $w$ is exactly $F_0/2^w$, although in fact it is a binomial random variable with
this number as mean\footnote{An exact value of $p$ is $(1 - \frac{1}{r \cdot 2^w})^{F_0}$,
which considers that any of $F_0$ $k$-mers may cause a collision at level $w$.
The approximate value of $p$ is almost equal to the exact value for $r, F_0$ used, however,
and the calculations with the approximate $p$ are much simpler.}.

The expected number of empty counters at level $w$ is thus $r \cdot p$. Let us denote the
number of observed empty counters at level $w$ as $t_0$. Using the asspumtion
$t_0 \approx r \cdot p$ we can easily derive the estimator of $F_0$:

$$ \hat F_0 = 2^w \cdot \frac{\ln(t_0/r)}{\ln\left(1 - \frac{1}{r}\right)} $$

To estimate the number of distinct $k$-mers, we choose one level of the sketch
$w^*$, so that the number of empty counters at this level ($t_0$) is closest to $r/2$.
In \cite{Melsted2014} it has been shown that selecting this level
provides a bounded error of $\hat F_0$ with guaranteed probability.

\paragraph{Estimator of $f_i$}
The expected number of distinct $k$-mers with abundance $i$ hashed to level $w$ is $f_i / 2^w$.
When a $k$-mer is hashed into level $w$, the probabilty that it is stored in a collision-free
counter is $(1 - \frac{1}{r})^{F_0/2^w - 1}$, which is the probability that no other $k$-mer 
from level $w$ will get hashed into the same counter. Thus we can estimate the number of
collision-free counters with value $i$ as  $f_i / 2^w \cdot (1 - \frac{1}{r})^{F_0/2^w - 1}$.
If we denote the number of observed collision-free counters with value $i$ as $t_i$,
we can derive the estimator of $f_i$:

$$ \hat f_i = t_i \cdot 2^w \cdot \left(1 - \frac{1}{r}\right)^{1 - F_0/2^w} $$

Again, one level $w^*$ is selected to estimate $f_i$, so that it maximizes $t_i$ -- the number 
of observed collision-free counters with value $i$. This decision was not discussed by the authors,
but it seems to be a reasonable choice to achieve the higest accuracy, since higher levels
would contain fewer counters with value $i$ and lower levels would countain fewer
collision-free counters.

\paragraph{Undetected collsions}
We use an assumption that if a counter holds a value $i$, it must originate from a $k$-mer 
with abundance $i$, but hashing collisions can occur. With the collision detection mechanism
in place, most of the counters with collisions are discarded, but some collisions can remain
undetected. The value $t_i$ is based on the number of non dirty counters, but these include both
true positives (collision-free counters) and false positives (counters with an undetected
collision). 

The authors have shown that the expected number of false positive at one level
is at most $r/u$ and that parameter $u$ can be set to make false positives negligible\footnote{We 
note that the expected number of false positives does not depend on
the number of distinct $k$-mers $F_0$. With increasing $F_0$ more collisions take place,
but also more collisions are detected and these two effects cancel each other out. 
The proof can be found in Lemma 4 of Appendix D of \cite{Sivadasan2016}.}. As with $2k$ 
bits per counter we would be able to retain the whole $k$-mer stored in this 
counter and thus detect all collisions perfectly, the parameter $u$ only provides 
an effective trade-off between memory usage and accuracy.

In our analysis we will ignore the effect of false positives and we will use an assumption
that all collisions are being detected.

\paragraph{Median amplification}
To further decrease the variance of estimates and to make use of multiprocessing, 
$t$ independent instances of Kmerlight's sketch are run concurrently.
Estimate $\hat F_0$ is then selected as median of $\hat F_0^{(1)}, \dots, \hat F_0^{(t)}$, 
and estimates of $f_i$ are also selected as medians from $t$ instances. 

\paragraph{Accuracy and complexity}
The parameters $r$, $u$ ant $t$ can be altered to achieve a viable memory-accuracy trade-off.
The algorithm uses $O(t \cdot r \cdot \log(F_0))$ memory words by $t$ instances of sketches
with $W = \log F_0$ levels with $r$ counters each. 
An update of $t$ sketches (processing of one $k$-mer) requires $O(t)$ time.

The authors have shown that the algorithm computes estimates $\hat F_0$ and $\hat f_i$
for sufficiently large $f_i$ ($f_i \geq F_0 / \lambda$) with a bounded relative error 
$(1-\eps)F_0 \leq \hat F_0 \leq (1+\eps)F_0,~ (1-\eps)f_i \leq \hat f_i \leq (1+\eps)f_i$ with
probability at least $1 - \delta$, when the parameters are set as follows: 
$t = O(\log(\lambda/\delta))$, $r = O(\frac{\lambda}{\eps^2})$
and $u = O(\frac{\lambda F_0}{\eps^2})$. 

Due to the loose constants in the asymptotic estimate, these values of $t, r, u$ cannot be
directly used in practice to guarantee the error bounds. The accuracy of this algorithm was tested
experimentally with arbitrary parameters $t=7, r=2^{16}, 2^{18}, u=2^{13}$.
