\section{Choice of Kmerlight's parameters}
\label{sec:parameters-choice}

As we have noted in the first section of this chapter (\ref{sec:kmerlight})
the authors of the original paper about Kmerlight \cite{Sivadasan2016} have
provided us with theoretical bounds of Kmerlight's errors in the following form:
Kmerlight computes estimates $\hat f_i$ (and $\hat F_0$) for sufficiently large
$f_i$ ($f_i \geq F_0 / \lambda$) with an error $(1-\eps)f_i \leq \hat f_i \leq (1+\eps)f_i$
with probability at least $1 - \delta$.

It is unclear, however, how to set the parameters $r$ (the number of counters at one level) 
and $t$ (the number of sketch instances from which a median will be selected).

In this section we use the approximate distribution of Kmerlight's errors (derived in the
previous section \ref{sec:estimation-variance}) to set the parameters in order to achieve
desired error bounded by $\eps$ with probability $1-\delta$.

\paragraph{Prediction/confidence interval} To gain the error bounds
in the aforementioned form we derive a prediction interval of $\hat f_i$. 

By standardization we get $(\hat f_i - f_i)/\sigma \,\dot\sim\, N(0,1)$ and so it holds that
$$P\left[-u\left(\frac{\alpha}{2}\right) \leq 
\frac{\hat f_i - f_i}{\sigma} \leq 
u\left(\frac{\alpha}{2}\right) \right] \approx 1 - \alpha$$
where $u$ is the critical value of the normal distribution. This formula can be simply
manipulated to yield the bounds for $\hat f_i$:
$$P\left[\left(1-u\left(\frac{\alpha}{2}\right)\sigma/f_i\right) f_i \leq \hat f_i \leq \left(1-u\left(\frac{\alpha}{2}\right)\sigma/f_i\right) f_i \right] \approx 1 - \alpha$$
From this form we can set $\eps = u\left(\frac{\alpha}{2}\right)\sigma/f_i$.

\paragraph{Simultaneous intervals} Since we would like to compute the estimates
for multiple ($m$) values of $f_i$, to achieve the bounded error with probability
$1 - \delta$, we set $\alpha = \frac{\delta}{m}$ following Bonferroni correction.

\medskip

Let us finish this section by describing the method that can be used to choose the suitable parameters for Kmerlight.

Standard deviation $\sigma$ can be calculated by the equation (\ref{eq:variance})
using $F_0, f_i, r, t$. Note that $\sigma$ depends logarithmically on $F_0$
and thus a very rough estimate of $F_0$ should be sufficient for parameter setting. 
We can demand the bounded precision only for sufficiently large $f_i$ by setting 
$f_i = F_0/\lambda$ ($\lambda=1000$ for example). The estimates of larger $f_i$ will be more precise.

Using the relation $\eps = u\left(\frac{\alpha}{2}\right)\sigma/f_i$ we can now directly
calculate $\eps$ for a desired probability $\delta$ and the parameters $r, t, \lambda$.
This result allows us to tweak the parameters $r, t$ to achieve the desired precision
without the need to actually run any computationally demanding program.
