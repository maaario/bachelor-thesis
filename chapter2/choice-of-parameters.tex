\section{Choice of Kmerlight's parameters}
\label{sec:parameters-choice}

As we have noted in section \ref{sec:kmerlight},
the authors of the original paper about Kmerlight \cite{Sivadasan2016} have
provided us theoretical bounds of Kmerlight's errors in the following form:
Kmerlight computes estimates $\hat f_i$ (and $\hat F_0$) for sufficiently large
$f_i$ ($f_i \geq F_0 / \lambda$) with an error $(1-\eps)f_i \leq \hat f_i \leq (1+\eps)f_i$
with probability at least $1 - \delta$.

It is unclear, however, how to set the parameters $r$ (the number of counters at one level) 
and $t$ (the number of sketch instances from which a median will be selected).

In this section we use the approximate distribution of Kmerlight's errors (derived in the
previous section) to set the parameters in order to achieve
desired error bounded by $\eps$ with probability $1-\delta$ (under the assumption that 
$\hat f_i$ is distributed according to our approximation; Kmerlight's authors do not make
such assumptions).

\paragraph{Prediction/confidence interval} To obtain the error bounds
in the aforementioned form, we derive a two-sided prediction interval of $\hat f_i$. 

By standardization we get $(\hat f_i - f_i)/\sigma \,\dot\sim\, N(0,1)$ and so it holds that
$$P\left[-u\left(\frac{\alpha}{2}\right) \leq 
\frac{\hat f_i - f_i}{\sigma} \leq 
u\left(\frac{\alpha}{2}\right) \right] \approx 1 - \alpha$$
where $u\left(\frac{\alpha}{2}\right)$ is the critical value of the normal distribution,
for significance level $\frac{\alpha}{2}$. This formula can be simply
manipulated to yield the bounds for $\hat f_i$:
$$P\left[\left(1-u\left(\frac{\alpha}{2}\right)\sigma/f_i\right) f_i \leq \hat f_i \leq \left(1-u\left(\frac{\alpha}{2}\right)\sigma/f_i\right) f_i \right] \approx 1 - \alpha$$
From this form we can see that error bound $\eps$ can be computed as
$\eps = u\left(\frac{\alpha}{2}\right)\sigma/f_i$.

\paragraph{Simultaneous intervals} Since we would like to compute the estimates
for multiple ($m$) values of $f_i$, to achieve the bounded error with probability at least
$1 - \delta$, we set $\alpha = \frac{\delta}{m}$ following Bonferroni correction.

\medskip

Let us finish this section by describing the method that can be used to choose the suitable
parameters $r, t$ for Kmerlight to produce estimates with relative errors bounded by $\eps$
with probability at least $1 - \delta$. 

Standard deviation $\sigma$ can be calculated by the equation (\ref{eq:variance})
using $F_0, f_i, r, t$. Note that $\sigma$ depends logarithmically on $F_0$
and thus a very rough estimate of $F_0$ should be sufficient for parameter setting. 
We can demand the bounded precision only for sufficiently large $f_i$ by setting 
$f_i = F_0/\lambda$ ($\lambda=1000$ for example). The estimates of larger $f_i$ will
be more precise. Finally, using the relation 
$\eps = u\left(\frac{\alpha}{2}\right)\sigma/f_i$ we can now directly calculate 
error bound $\eps$ for a given probability $\delta$ and the parameters $r, t, \lambda$.

This result allows us to explore various values of the parameters $r, t$
and their influence on approximation errors without the need to actually 
run any computationally demanding program. For example, with the use of procedure described above,
we can find the combination of parameters $r, t$, which minimizes the approximation errors
for a fixed memory limit.

